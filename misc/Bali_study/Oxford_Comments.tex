\documentclass[12pt]{article}
\usepackage[hmarginratio=1:1,
top=20mm,bottom=20mm, 
left=20mm, right=20mm, 
columnsep=20pt]{geometry}
\usepackage{quotes}
\usepackage{amsmath}
\usepackage{caption}
\usepackage{fancyhdr} 
\usepackage{hyperref}
%\usepackage{framed}
\usepackage{float}
\usepackage{setspace}
\hypersetup{colorlinks=true, 
linkcolor=blue, 
filecolor=blue, 
urlcolor=blue,
citecolor=black}
\pagestyle{fancy} 
\fancyfoot{}
\fancyhead[LO]{}
\fancyhead[CO]{}
\fancyhead[RO]{\thepage}
\usepackage{graphicx}
\begin{document}

\onehalfspacing

\noindent
Gagan Atreya \\
May 12, 2023 

\subsection*{Bali Bombings: Shared Memories, Identity Fusion, and Morality}

\begin{itemize}

\item \textbf{Missing values were replaced using multiple permutations.} 

Ideally, this section should be explained in greater detail. Why is multiple permutation the best appraoch here as opposed to other ways of dealing with missing values?  

\item \textbf{Sample size}

It seems like the sample size is really quite small to be able to justify the tests performed in the analysis. It's probably logistically difficult to recruit a larger sample, but some kind of power analysis to justify the sample size would be helpful. 

\item \textbf{Statistical tests}

Basically none or very few of the tests are significant. There is a section where it is stated the t-test is "just about significant" at p = 0.0498. That's not really a technically valid term (I don't even know what it means). Just looking at the barplots and considering the overall sample size in the data, the reasonable conclusion is that there really isn't any kind of significant differences between the conditions. This is also the case with the mediation analysis where the results are often contradictory. 

As the "Limitations" sections states, there are quite a few severe issues with the data in terms of missingness, sample size, lack of normality, etc for the results to be convincing. The statistical tests themselves are fine from what I can tell (although I have no access to the actual data and/or the reproducible code). However, I would conclude that the results are not very convincing beyond a few descriptive relationships. 

\end{itemize}








\end{document}